\documentclass[paper.tex]{subfiles}
\begin{document}

\begin{abstract}

  Many applications depend on hardware-supported floating point
  arithmetic to efficiently approximate computations over real
  numbers.  Unfortunately, such approximations introduce rounding
  error which can accumulate to produce unacceptable results.  When
  the largest hardware-supported precision does not provide sufficient
  accuracy, developers may be forced to simulate arbitrary precision
  floating point arithmetic in software, incurring orders of magnitude
  slowdown.  While the numerical methods literature provides
  techniques to mitigate rounding error without increasing precision,
  applying these techniques often requires deep expertise and
  substantial manual effort.

  We introduce \casio, an \textit{automated} approach to improving
  floating point accuracy by searching for error-reducing program
  transformations.  \casio's search is guided by techniques to
  estimate rounding error and its sources, as well as an approach to
  detect input regimes with distinct error behavior and select the
  best program variant for each regime.  We evaluate \casio on
  benchmarks drawn from numerical methods textbooks, mathematical
  libraries, and recent scientific articles, demonstrating that \casio
  substantially reduces error while imposing modest overhead.

\end{abstract}

% Furthermore, such incorrect results are difficult to detect and
% debug.

% ideal real number computation

% The numerical methods literature provides techniques to improve the
% numerical accuracy of real computations approximated in floating
% point.  Unfortunately, applying these techniques requires
% substantial expertise and manual effort: analysis + transformation.
% As more and more scientists write numerical programs, the lack of
% accuracy becomes increasingly troubling.  These non-expert
% programmers usually lack the training and time necessary to employ
% numerical methods in their code, and often may not even be aware of
% the subtle challenges this domain presents.

\end{document}