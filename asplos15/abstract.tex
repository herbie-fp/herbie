\documentclass[paper.tex]{subfiles}
\begin{document}

\begin{abstract}

  Many applications from science and engineering depend on floating
  point hardware to efficiently approximate computations over real
  numbers.  Unfortunately, such approximations introduce rounding
  error, which can accumulate to produce unacceptable results.  When
  the largest hardware-supported precision does not provide sufficient
  accuracy, developers may be forced to simulate arbitrary precision
  floating point arithmetic in software, incurring orders of magnitude
  slowdown.  While the numerical methods literature provides
  techniques to mitigate rounding error without increasing precision,
  these techniques often require both understanding the details of
  floating point arithmetic and also manually rearranging
  computations.

  We introduce \casio, which \textit{automatically} improves floating
  point accuracy by searching for error-reducing program
  transformations.  This search is guided by techniques for estimating
  rounding error and its sources, as well as mechanisms which detect
  input regimes with distinct error behavior and select the best
  program variant for each regime.  We evaluate \casio on examples
  from a classic numerical methods textbook and demonstrate that it
  can effectively reduce rounding error. We also consider \casio's
  broader applicability by improving the accuracy of expressions from
  a mathematical library as well as formulas from recent scientific
  articles.

\end{abstract}

\end{document}