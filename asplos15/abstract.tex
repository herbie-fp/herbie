\documentclass[paper.tex]{subfiles}
\begin{document}

\begin{abstract}

  Many applications from science and engineering depend on floating
  point hardware to efficiently approximate computations over real
  numbers.  Unfortunately, such approximations introduce rounding
  error, which can accumulate to produce unacceptable results.  When
  the largest hardware-supported precision does not provide sufficient
  accuracy, developers may be forced to simulate arbitrary precision
  floating point arithmetic in software, incurring orders of magnitude
  slowdown.  While the numerical methods literature provides
  techniques to mitigate rounding error without increasing precision,
  these techniques often require both understanding the details of
  floating point arithmetic and also manually rearranging
  computations.

  We introduce \casio, which automatically improves floating point
  accuracy by searching for error-reducing program transformations.
  This search is guided by techniques for estimating rounding error,
  finding its sources, and dividing the input space into regimes with
  distinct error behavior, allowing \casio to select the best program
  variant for each regime.  We evaluate \casio on examples from a
  classic numerical methods textbook and find that it reduces average
  error by \todo{some quantitative measure}. We also present a series
  of case studies using a formulas found in the wild to show that
  \casio has broad applicability.

\end{abstract}

\end{document}