\documentclass[paper.tex]{subfiles}
\begin{document}

\begin{abstract}

  Many applications from science and engineering depend on floating
  point arithmetic implemented in hardware to efficiently approximate
  computations over real numbers.  Unfortunately, such approximations
  introduce rounding error which can accumulate to produce
  unacceptable results.  When the largest hardware-supported precision
  does not provide sufficient accuracy, developers may be forced to
  simulate arbitrary precision floating point arithmetic in software,
  incurring orders of magnitude slowdown.  While the numerical methods
  literature provides techniques to mitigate rounding error without
  increasing precision, applying these techniques often requires deep
  expertise and substantial manual effort.

  We introduce \casio, an \textit{automated} approach to improving
  floating point accuracy by searching for error-reducing program
  transformations.  \casio's search is guided by techniques to
  estimate rounding error and its sources, as well as an approach to
  detect input regimes with distinct error behavior and select the
  best program variant for each regime.  We evaluate \casio on
  benchmarks drawn from numerical methods textbooks and consider its
  broader applicability to a mathematical library as well as formulas
  drawn from recent scientific articles.  Our results demonstrate that
  \casio can substantially reduce error.

\end{abstract}

\end{document}