\documentclass[paper.tex]{subfiles}
\begin{document}

\section{Evaluation}

\todo{compiler flags order of magnitude speed up !!!}

In addition to the previously discussed examples,
we evaluated \casio on benchmarks drawn from
numerical methods textbooks,
physics papers from Physical Review A,
and standard definitions of mathematical functions from several sources,
including existing mathematic libraries.
\todo{reference results}
Beginning from the naive implementation as an AST,
\casio regularly finds vast improvements behavior for those expressions which have high error.
In many cases, the accuracy of the programs outputted by casio 
exceeds that of rewrites provided by numerical methods textbooks.
In many cases, \casio discovers rewrites of a complexity 
such that a human would be unlikely to discover rewrites of equivilent accuracy.

\subsection{Numerical Methods for Scientists and Engineers}

To test the effectiveness of the search done by \casio,
we took the numerical methods examples from
Numerical Methods for Scientists and Engineers (NMSE) chapter 3.
Some of these examples also included suggested solution,
against which we compared the results of our search.
In many cases, \casio is able to programs which have
similar error behavior to the suggested solutions,
and in some cases \casio is able to surpass the
example solutions, resulting in a program with even less error.

\subsection{Physical Review A}

As part of our benchmarks, we searched through Physical Review A Volume 89
and handpicked formulas which appeared as though
their naive, direct implementation would have high error.
Our set is by no means comprehensive,
as the content we would need to review is significantly large,
but our results show that someone wishing to replicate the results of these papers
would face significant numerical challenges that \casio can help mitigate,
if they were to implement the formulas directly as written.

\subsection{Standard Mathematical Functions}

We acquired benchmarks based on standard mathematical functions from two sources.
Firstly, \todo{tell them we used javascript libraries without telling them we used javascript libraries.}.
Although this code was likely not analyzed for numerical accuracy,
and even in some cases cites sources such as wikipedia for their formulas,
this code is significant in that it is an example of ``real-world code.''
Secondly, we took the definitions of the hyperbolic functions from their mathematical definitions.
These code snippets are significant in that
they represent the naive approach that a programmer might take
when attempting to implement these functions in their own code,
and would be able to safely use if they were using a tool like casio
in their code pipeline.

\end{document}
